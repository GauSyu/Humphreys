\part{Isomorphism and Conjugacy Theorem}
\section{Isomorphism theorem}

\begin{ex}
  Generalize Theorem 14.2 to the case: $L$ semisimple.
\end{ex}

\begin{ex}
  Let $L = \sl(2, F)$. If $H,H'$ are any two maximal toral subalgebras of $L$, prove that there exists an automorphism of $L$ mapping $H$ onto $H'$.
\end{ex}

\begin{ex}
  Prove that the subspace $M$ of $L \times L'$ introduced in the proof of Theorem 14.2 will actually equal $D$, if $x$ and $x'$ are chosen carefully.
\end{ex}

\begin{ex}
  Let $\sigma$ be as in Proposition 14.3. Is it necessarily true that $\sigma(x_{\alpha}) = -y_{\alpha}$ for nonsimple $\alpha$, where $[x_{\alpha},y_{\alpha}] = h_{\alpha}$?
\end{ex}

\begin{ex}
  Consider the simple algebra $\sl(3,F)$ of type $A_2$. Show that the subgroup of $\Int L$ generated by the automorphisms $\tau_{\alpha}$ in (14.3) is strictly larger than the Weyl group (here $\Sss_3$). [View $\Int L$ as a matrix group and compute $\tau_{\alpha}^2$ explicitly.]
\end{ex}

\begin{ex}
  Use Theorem 14.2 to construct a subgroup $\Gamma(L)$ of $\Aut L$ isomorphic to the group of all graph automorphisms (12.2) of $\Phi$.
\end{ex}

\begin{ex}
  For each classical algebra (1.2), show how to choose elements $h_{\alpha}\in H$ corresponding to a base of $\Phi$ (cf. Exercise \ref{8.2}).
\end{ex}

\section{Cartan subalgebras}
Cartan subalgebra will be abbreviated as CSA.

\begin{ex}
  A semisimple element of $\sl(n, F)$ is regular if and only if its eigenvalues are all distinct (i.e., if and only if its minimal and characteristic polynomials coincide).
\end{ex}

\begin{ex}
  Let $L$ be semisimple ($\Char F = 0$). Deduce from Exercise \ref{8.7} that the only solvable Engel subalgebras of $L$ are the CSA's.
\end{ex}

\begin{ex}
  Let $L$ be semisimple ($\Char F = 0$), $x \in L$ semisimple. Prove that $x$ is regular if and only if $x$ lies in exactly one CSA.
\end{ex}

\begin{ex}
  Let $H$ be a CSA of a Lie algebra $L$. Prove that $H$ is maximal nilpotent, i.e., not properly included in any nilpotent subalgebra of $L$. Show that the converse is false.
\end{ex}

\begin{ex}
  Show how to carry out the proof of Lemma A of (15.2) if the field $F$ is only required to be of cardinality exceeding $\dim L$.
\end{ex}

\begin{ex}
  Let $L$ be semisimple ($\Char F = 0$), $L'$ a semisimple subalgebra. Prove that each CSA of $L'$ lies in some CSA of $L$. [Cf. Exercise \ref{6.9}.]
\end{ex}

\section{Conjugacy theorems}
\begin{ex}
  Prove that $\Eee(L)$ has order one if and only if $L$ is nilpotent.
\end{ex}

\begin{ex}\label{16.2}
  Let $L$ be semisimple, $H$ a CSA, $\Delta$ a base of $\Phi$. Prove that any subalgebra of $L$ consisting of nilpotent elements, and maximal with respect to this property, is conjugate under $\Eee(L)$ to $N(\Delta)$, the derived algebra of $B(\Delta)$.
\end{ex}

\begin{ex}
  Let $\Psi$ be a set of roots which is \termin{closed root set}[closed] $(\alpha, \beta \in\Psi, \alpha+\beta \in\Phi$ implies $\alpha+\beta\in\Psi$) and satisfies $\Psi\cap -\Psi=\varnothing$.
  Prove that $\Psi$ is included in the set of positive roots relative to some base of $\Phi$. [Use Exercise \ref{16.2}.] (This exercise
belongs to the theory of root systems, but is easier to do using Lie algebras.)
\end{ex}

\begin{ex}
  How does the proof of Theorem 16.4 simplify in case $L = \sl(2, F)$?
\end{ex}

\begin{ex}
  Let $L$ be semisimple. If a semisimple element of $L$ is regular, then it lies in only finitely many Borel subalgebras. (The converse is also true, but harder to prove, and suggests a notion of ``regular'' for elements of $L$ which are not necessarily semisimple.)
\end{ex}