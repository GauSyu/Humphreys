\part{Existence Theorem}
\section{Universal enveloping algebras}

\begin{ex}
  Prove that if $\dim L < \infty$, then $\UUu(L)$ has no zero divisors. [Hint: Use the fact that the associated graded algebra $\GGg$ is isomorphic to a polynomial algebra.]
\end{ex}
\begin{proof}
	Let $A$ be a filtered algebra and $B$ its associated graded algebra ($B_i=A_i/A_{i-1}$). Then, if $B$ is integral, then so is $A$. Otherwise, let $xy = 0$ in $A$ with $x\in A_n\setminus A_{n-1}$, $y\in A_m\setminus A_{m-1}$ and $\bar{x},\bar{y}$ their images in $B$. Then $\bar{x}\bar{y}=0$. But $B$ is integral, so either $\bar{x}=0$ or $\bar{y}=0$, which means either $x\in A_{n-1}$ or $y\in A_{m-1}$, a contradiction. In our case, $\GGg$ is isomorphic to a polynomial algebra, hence integral.
\end{proof}

\begin{ex}
	Let $L$ be the two dimensional nonabelian Lie algebra (1.4), with $[x,y] = x$. Prove directly that $i\colon L\to \UUu(L)$ is injective (i.e., that $J\cap L = 0$).
\end{ex}
\begin{proof}
	In this case, $J$ is generated by $x\otimes y - y\otimes x - x$. Therefore, $J\cap L = 0$.
\end{proof}

\begin{ex}
	If $x\in L$, extend $\ad x$ to an endomorphism of $\UUu(L)$ by defining $\ad x(y) = xy-yx (y\in \UUu(L))$. If $\dim L < \infty$, prove that each element of $\UUu(L)$ lies in a finite dimensional L-submodule. [If $x,x_1,\cdots,x_m\in L$, verify that 
	\[\ad x(x_1\cdots x_m) = \sum_{i=1}^m x_1x_2\cdots\ad x(x_i)\cdots x_m.\]
\end{ex}
\begin{proof}
	Direct computation shows
	\begin{align*}
		\sum_{i=1}^m x_1x_2\cdots\ad x(x_i)\cdots x_m &
		= \sum_{i=1}^m x_1x_2\cdots(xx_i-x_ix)\cdots x_m \\
		&= xx_1x_2\cdots x_m + \sum_{i=2}^m x_1\cdots xx_i\cdots x_m \\
		&\quad - \sum_{i=1}^{m-1} x_1\cdots x_ix\cdots x_m - x_1x_2\cdots x_mx \\
		&= xx_1x_2\cdots x_m - x_1x_2\cdots x_mx \\
		&= \ad x(x_1x_2\cdots x_m).
	\end{align*}
\end{proof}

\begin{ex}
	If $L$ is a free Lie algebra on a set $X$, prove that $\UUu(L)$ is isomorphic to the tensor algebra on a vector space having $X$ as basis.
\end{ex}
\begin{proof}
	Let $V$ be the vector space spanned by $X$ and $\TTt(V)$ the tensor algebra of $V$. Consider the following diagram where $v,l,i,j$ are the canonical inclusions and other are constructed as follows.
	\[
	\xymatrix@1{
		X\ar[d]_-{v}\ar[r]^-{l} & L\ar[d]_-{\lambda}\ar[r]^-{i} & \UUu(L)\ar[dl]_-{\phi} \\
		V\ar[r]_-{j}\ar[ur]^-{\tau} & \TTt(V)\ar@/_/[ur]_-{\psi} &
	}
	\]
	
	\begin{enumerate}
		\item By the universal property of $V$, there exists a unique linear map $\tau\colon V\to L$ such that $\tau\circ v = l$;
		\item By the universal property of $L$, there exists a unique Lie algebra homomorphism $\lambda\colon L\to\TTt(V)$ such that $\lambda\circ\tau = j$;
		\item By the universal property of $\UUu(L)$, there exists a unique algebra homomorphism $\phi\colon\UUu(L)\to\TTt(V)$ such that $\phi\circ i = \lambda$;
		\item By the universal property of $\TTt(V)$, there exists a unique algebra homomorphism $\psi\colon\TTt(V)\to\UUu(L)$ such that $\psi\circ j = i\circ\tau$.
	\end{enumerate}
	
	By the universal property of $\UUu(L)$ and $\TTt(V)$, one can see that $\phi\circ\psi = \id$ and $\psi\circ\phi =\id$. Therefore, they are isomorphic.
\end{proof}

\begin{ex}
	Describe the free Lie algebra on a set $X = \{x\}$.
\end{ex}
\begin{proof}
	Let $V$ be the vector space spanned by $X$ and $\TTt(V)$ the tensor algebra of $V$. In this case, we have $\TTt(V) \cong F[x]$. The free Lie algebra $L$ generated by $X$ is the Lie subalgebra of $\TTt(V)$ generated by $X$. Thus $L=V$ equipped with trivial bracket.
\end{proof}

\begin{ex}
	How is the PBW theorem used in the construction of free Lie algebras?
\end{ex}
\begin{proof}
	Let $L$ be the free Lie algebra generated by $X$. Let $M$ be an arbitrary Lie algebra and $f\colon X\to M$ a map. Then consider the following diagram.
	\[
	\xymatrix@1{
		X\ar[r]^-{l}\ar[dr]_-{f} & L\ar[r]^-{\lambda}\ar@{-->}[d]^-{f'} & \TTt(V)\ar[d]^-{g} \\
		& M\ar[r]^-{i} & \UUu(M)
		}
	\]
	
	By the universal property of $\TTt(V)$, there exists a unique algebra homomorphism $g$ making the diagram commute. Restrict $g$ to $L$, we get a Lie algebra homomorphism $f'$ fitting the commutative diagram. Since $\lambda\colon L\to \TTt(V)$ is injective, to guarantee the uniqueness of $f'$, one need the fact that $i\colon M\to \UUu(M)$ is injective, which follows from PBW.
\end{proof}

\section{Generators and relations}

\begin{ex}
	Using the representation of $L_0$ on $V$ (Proposition 18.2), prove that the algebras $X, Y$ described in Theorem 18.2 are (respectively) free Lie algebras on the sets of $x_i, y_i$.
\end{ex}
\begin{proof}
	For $Y$: the restriction of the action of $L_0$ on $V$ to $Y$ is isomorphic to its left multiplication. In this way, we can identify $Y$ as the Lie subalgebra of the tensor algebra $V$, which is generated by $v_1,\cdots,v_l$. Thus $Y$ is the free Lie algebra on $y_1,\cdots,y_l$.
	
	By the duality of $X$ and $Y$, the conclusion follows.
\end{proof}

\begin{ex}
	When $\rank \Phi = 1$, the relations $(S_{ij}^+), (S_{ij}^−)$ are vacuous, so $L_0 = L \cong \sl(2,F)$.By suitably modifying the basis of $V$ in (18.2), show that $V$ is isomorphic to the module $Z(0)$ constructed in Exercise \ref{7.7}.
\end{ex}
\begin{proof}
	In this case, $V=F[v]$. So, we put $w_i = \frac{1}{i!}v^i$. Then these $w_i$ form a basis of $V$. Direct computation shows:
	\begin{align*}
		h.w_i & = \frac{1}{i!}h.v^i = \frac{-2i}{i!}v^i = -2iw_i, \\
		y.w_i & = \frac{1}{i!}y.v^i = \frac{1}{i!}v^{i+1} = (i+1)w_{i+1}, \\
		x.w_i & = \frac{1}{i!}x.v^i = \frac{1}{i!}(vx.v^{i-1}-2(i-1)v^{i-1}) \\
		& = \frac{-i(i-1)}{i!}v^{i-1} = -(i-1)w_{i-1}.
	\end{align*}
	Therefore $v^i\mapsto w_i$ gives the isomorphism $V\cong Z(0)$.
\end{proof}

\begin{ex}
	Prove that the ideal $K$ of $L_0$ in (18.3) lies in every ideal of $L_0$ having finite codimension (i.e., $L$ is the largest finite dimensional quotient of $L_0$).
\end{ex}
\begin{proof}
	Let $I$ be an ideal of $L_0$ having finite codimension. If $I$ doesn't contain some $x_{ij}$. Then, by apply $\ad x_i$ to the image of $x_j$ in $L_0/I$, we get an infinite dimensional space, which is a contradiction.
\end{proof}

\begin{ex}
	Prove that each inclusion of Dynkin diagrams (e.g. $E_6\subset E_7\subset E_8$) induces a natural inclusion of the corresponding semisimple Lie algebras.
\end{ex}
\begin{proof}
	By Exercise \ref{11.6}, each inclusion of Dynkin diagrams induces an inclusion of the corresponding root systems $f\colon\Phi\to\Phi'$ and hence an isomorphism from $\Phi$ to its image. Then, the Serre theorem tells that this induces an isomorphism from the semisimple Lie algebra corresponding to $\Phi$ to that corresponding to $f(\Phi)$, which is a subalgebra of the semisimple Lie algebra corresponding to $\Phi'$. This gives an inclusion of Lie algebras.
\end{proof}


\section{The simple algebras}
