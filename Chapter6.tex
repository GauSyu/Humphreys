\part{Representation Theory}
\section{Weights and maximal vectors}
\begin{ex}
	If $V$ is an arbitrary $L-$module, then the sum of its weight spaces is direct.
\end{ex}
\begin{proof}
	Let $\lambda,\mu$ be two distinct weights and $v\in V_{\lambda}\cap V_{\mu}$. Then, for all $h\in H$, we have $h.v=\lambda(h)v=\mu(h)v$. Since $\lambda\neq\mu$, there exists some $h\in H$ such that $\lambda(h)\neq\mu(h)$. Thus $v=0$.
\end{proof}
\begin{ex}
	\begin{enumerate}[(a)]
		\item If $V$ is an irreducible $L-$module having at least one (nonzero) weight space, prove that $V$ is the direct sum of its weight spaces.
		\item Let $V$ be an irreducible $L-$module. Then $V$ has a (nonzero) weight space if and only if $\UUu(H).v$ is finite dimensional for all $v\in V$, or if and only if $\AAa.v$ is finite dimensional for all $v\in V$ (where $\AAa$ = subalgebra with $1$ generated by an arbitrary $h\in H$ in $\UUu(H)$).
		\item Let $L = \sl(2, F)$, with standard basis $(x, y, h)$. Show that $1-x$ is not invertible in $\UUu(L)$, hence lies in a maximal left ideal $I$ of $\UUu(L)$. Set $V = \UUu(L)/I$, so $V$ is an irreducible $L-$module. Prove that the images of $1, h, h^2, \cdots$ are all linearly independent in $V$ (so $\dim V = \infty$), using the fact that
		\[
		(x-1)^rh^s \equiv 
		\begin{cases}
		0\mod I, & r>s \\
		(-2)^rr!\cdot1 \mod I, & r=s.
		\end{cases}
		\]
		Conclude that $V$ has no (nonzero) weight space.
	\end{enumerate}
\end{ex}
\begin{proof}
	(a) follows from Lemma 20.1(b).
	
	(b): Let $v\in V$ and write $v=\sum v_{\lambda}$ where $\lambda$ are some weights and $v_{\lambda}\in V_{\lambda}$. Then $\AAa.v \subset \bigoplus V_{\lambda}$ and hence is finite dimensional. 
	
	Conversely, if $\AAa.v$ is finite dimensional for all $v\in V$, then there exists $n$ such that $v,h.v,\cdots,h^{n-1}.v$ are linearly independent while $v, h.v, \cdots, h^{n}.v$ are not. If $n=1$, then $v$ is an eigenvector of $h$ and hence $H$ and therefore $V$ has a (nonzero) weight space. If $n>1$, let $f(x)$ be the monic polynomial of degree $n$ such that $f(h).v=0$ and $\alpha$ a root of $f(x)$. Write $f(x)=(x-\alpha)g(x)$ and let $v'=h.v-\alpha v$. Then we have
	\[
	g(h).v' = g(h).((h-\alpha).v) = f(h).v = 0.
	\]
	Repeat this process, we sill finally find a $v_0$ such that $v_0, h.v_0$ are linearly dependent. As we have shown this implies the existence of (nonzero) weight space.
\end{proof}
\begin{ex}
	Describe weights and maximal vectors for the natural representations of the linear Lie algebras of types $A_l - D_l$ described in (1.2).
\end{ex}
\begin{ex}
	Let $L = \sl(2, F), \lambda\in H^{\ast}$. Prove that the module $Z(\lambda)$ for $\lambda=\lambda(h)$ constructed in Exercise \ref{7.7} is isomorphic to the module $Z(\lambda)$ constructed in (20.3). Deduce that $\dim V(\lambda) < \infty$ if and only if $\lambda(h)$ is a nonnegative integer.
\end{ex}
\begin{proof}
	Let $w_i = \frac{y^i}{i!}\otimes v^+$. Then direct computation shows
	\begin{align*}
		yw_i & = (i+1)w_{i+1}; \\
		hw_i & = \frac{hy^i}{i!}\otimes v^+ \\
		         & = \frac{([h,y]+yh)y^{i-1}}{i!}\otimes v^+ \\
		         & = \frac{1}{i!}(-2y^i\otimes v^+ + y(hy^{i-1}\otimes v^+)) \\
		         & = \frac{1}{i!}(-2y^i\otimes v^+ + (i-1)!yhw_{i-1}) \\
		         & = \frac{1}{i!}(-2y^i\otimes v^+ + y(-2y^{i-1}\otimes v^+ + (i-2)!yhw_{i-2})) \\
		         & \cdots\qquad\cdots \\
		         & = \frac{1}{i!}(-2iy^i\otimes v^+ + y^ih\otimes v^+)  \\
		         & = \frac{1}{i!}(-2iy^i\otimes v^+ + y^i\otimes \lambda v^+) \\
		         & = (\lambda-2i)w_i; \\
		xw_i & = \frac{xy^i}{i!}\otimes v^+ \\
		         & = \frac{([x,y]+yx)y^{i-1}}{i!}\otimes v^+ \\
		         & = \frac{1}{i}(hw_{i-1} + yxw_{i-1}) \\
		         & = \frac{1}{i}((\lambda-2i+2)w_{i-1}+y\frac{1}{i-1}(hw_{i-2} + yxw_{i-2})) \\
		         & = \cdots\qquad\cdots \\
		         & = \frac{1}{i}((\lambda-2i+2)w_{i-1}+(\lambda-2(i-2))w_{i-1}+\cdots+\lambda w_{i-1}+ \frac{1}{(i-1)!}y^ixw_{0}) \\
		         & = (\lambda-i+1)w_{i-1}.
	\end{align*}
	
	Therefore $\frac{y^i}{i!}\otimes v^+\mapsto w_i$ gives the required isomorphism.
\end{proof}

\begin{ex}
	If $\mu\in H^{\ast}$, define $\Pp(\mu)$ to be the number of distinct sets of nonnegative integers $k_{\alpha} (\alpha\succ0)$ for which $\mu =\sum_{\alpha\succ0} k_{\alpha}\alpha$. Prove that $\dim Z(\lambda)_\mu = \Pp(\lambda-\mu)$, by describing a basis for $Z(\lambda)_\mu$.
\end{ex}
\begin{proof}
	Let $\lambda-\mu = \sum_{\alpha\succ0} k_{\alpha}\alpha$. By Theorem 20.2(b), $(\prod_{\alpha\succ0} y_{\alpha}^{k_{\alpha}}).v^+$ spans $Z(\lambda)_{\mu}$. By PBW, they are linearly independent.
\end{proof}

\begin{ex}
	Prove that the left ideal $I(\lambda)$ introduced in (20.3) is already generated by the elements $x_{\alpha}, h_{\alpha}-\lambda(h_{\alpha}).1$ for $\alpha$ simple.
\end{ex}
\begin{proof}
	By Serre relations, we can construct all $x_{\alpha}, h_{\alpha}-\lambda(h_{\alpha}).1$ from those for simple $\alpha$.
\end{proof}


%\section{Finite dimensional modules}

%\section{Multiplicity formula}

%\section{Characters}

%\section{Formulas of Weyl, Kostant, and Steinberg}