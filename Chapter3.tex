\part{Root System}
\section{Axiomatics}
Unless otherwise specified, $\Phi$ denotes a root system in $E$, with Weyl group $\Ww$.

\begin{ex}
  Let $E'$ be a subspace of $E$. If a reflection $\sigma_{\alpha}$ leaves $E'$ invariant, prove that either $\alpha \in E'$ or else $E' \subset P_{\alpha}$.
\end{ex}
\begin{proof}
  Suppose $E' \not\subset P_{\alpha}$. Let $\lambda\in E'\setminus P_{\alpha}$, then $\sigma_{\alpha}(\lambda) = \lambda - <\lambda,\alpha> \alpha \in E'$. Since $\lambda\not\in P_{\alpha}$, $<\lambda,\alpha>\neq0$. Hence $\alpha\in E'$.
\end{proof}

\begin{ex}\label{9.2}
  Prove that $\codual{\Phi}$ is a root system in $E$, whose Weyl group is naturally isomorphic to $\Ww$; show also that $< \codual{\alpha}, \codual{\beta}>=<\beta,\alpha>$, and draw a picture of $\codual{\Phi}$ in the cases $A_1,A_2,B_2,G_2$.
\end{ex}
\begin{proof}
  By definition, $\codual{\Phi}$ consists of $\codual{\alpha} = 2\alpha/(\alpha,\alpha)$ for $\alpha\in\Phi$. If $\Phi$ is finite, spans $E$, and does not contain $0$, it is obvious that $\codual{\Phi}$ also satisfies these conditions. Since $\codual{(c\alpha)}=c\codual{(\alpha)}$ for $c\in \RR$, it follows that the only multiples of $\codual{\alpha}$ in $\codual{\Phi}$ are $\pm\codual{\alpha}$. Also, for $\alpha,\beta\in\Phi$,
  \begin{equation*}
    \sigma_{\codual{\alpha}}\codual{\beta} = \codual{\beta} - \frac{2(\codual{\beta},\codual{\alpha})}{(\codual{\alpha},\codual{\alpha})}\codual{\alpha} = \frac{2\beta}{(\beta,\beta)} - \frac{4(\beta,\alpha)}{(\beta,\beta)(\alpha,\alpha)}\alpha
  \end{equation*}
  and
  \begin{equation*}
    \codual{(\sigma_{\alpha}(\beta))} = \codual{\left(\beta-\frac{2(\beta,\alpha)}{(\alpha,\alpha)}\alpha\right)} = \frac{2\beta-4\frac{(\beta,\alpha)}{(\alpha,\alpha)}\alpha}{(\beta,\beta)-4\frac{(\beta,\alpha)^2}{(\alpha,\alpha)}+44\frac{(\beta,\alpha)^2}{(\alpha,\alpha)}} = \frac{2\beta}{(\beta,\beta)} - \frac{4(\beta,\alpha)}{(\beta,\beta)(\alpha,\alpha)}\alpha
  \end{equation*}
  so $\sigma_{\codual{\alpha}}\codual{\beta} = \codual{(\sigma_{\alpha}(\beta))}$, and hence $\codual{\Phi}$ is invariant under $\sigma_{\codual{\alpha}}$. Finally,
  \begin{equation*}
    <\codual{\alpha},\codual{\beta}> = \frac{2(\codual{\alpha},\codual{\beta})}{(\codual{\beta},\codual{\beta})} = \frac{2(\alpha,\beta)}{(\alpha,\alpha)} = <\beta,\alpha> \in\ZZ
  \end{equation*}
  so all of the axioms for a root system are satisfied for $\codual{\Phi}$.

  Finally, by the above calculations, the bijection $\alpha\mapsto\codual{\alpha}$ induces an isomorphism $\sigma_{\alpha}\mapsto\sigma_{\codual{\alpha}}$ (thinking of the Weyl groups as subgroups of the symmetric group on $\Phi$ and $\codual{\Phi}$).
\end{proof}

\begin{ex}\label{9.3}
  In Table 1, show that the order of $\sigma_{\alpha}\sigma_{\beta}$ in $\Ww$ is (respectively) $2,3,4,6$ when $\theta=\pi/2, \pi/3$ (or $2\pi/3$), $\pi/4$ (or $3\pi/4$), $\pi/6$ (or $5\pi/6$). [Note that $\sigma_{\alpha}\sigma_{\beta}$ = rotation through $2\theta$.]
\end{ex}

\begin{ex}
  Prove that the respective Weyl groups of $A_1 \times A_1, A_2, B_2, G_2$ are dihedral of order $4,6,8,12$. If $\Phi$ is any root system of rank $2$, prove that its Weyl group must be one of these.
\end{ex}
\begin{proof}
  $\sigma_{\alpha}$ and $\sigma_{\alpha}\sigma_{\beta}$ generate the Weyl group, then the conclusions follow from Exercise \ref{9.3}.
\end{proof}
\begin{proof}
  There are only two reflections for the Weyl group of $A_1\times A_1$, and they commute with each other, so its Weyl group is isomorphic to $Z/2\times Z/2$, which is the dihedral group of order $4$.

  Picking alternating chambers of $A_2$, draw a regular triangle. The reflections of $A_2$ are symmetries of this triangle, so its Weyl group is $\Sss_3$, the dihedral group of order $6$.

  Drawing a square with vertices on the diagonal vectors of $B_2$, we see that all reflections preserve this square, and since the symmetries of the square are generated by reflections, the Weyl group of $B_2$ is $D_4$, the dihedral group of order $8$.

  Finally, the reflections of $G_2$ preserve a regular hexagon whose vertices are on the short vectors. So the Weyl group of $G_2$ is a subgroup of $D_6$, but since the reflections generate it, we get the whole group.

  The fact that the Weyl group of every rank $2$ root system must be dihedral of order $4, 6, 8$ or $12$ follows from the possibilities allowed in Table $1$.
\end{proof}

\begin{ex}
  Show by example that $\alpha-\beta$ may be a root even when $(\alpha,\beta) \leqslant 0$ (cf. Lemma 9.4).
\end{ex}
\begin{proof}
  This can be seen in the root system $G_2$. Using the labels in Figure 9.1, we have that $\alpha$ and $\alpha+\beta$ form an obtuse angle, i.e., $(\alpha,\alpha+\beta)\leqslant0$, but that $\beta$ is a root.
\end{proof}

\begin{ex}
  Prove that $\Ww$ is a normal subgroup of $\Aut\Phi$ (= group of all isomorphisms of $\Phi$ onto itself).
\end{ex}
\begin{proof}
  Any element of $\Ww$ can be written $\sigma_{\alpha_1}\cdots\sigma_{\alpha_r}$ for $\alpha_i\in\Phi$. Then for $\tau\in\Aut\Phi$, we have
  \begin{equation*}
    \tau(\sigma_{\alpha_1}\cdots\sigma_{\alpha_r})\tau^{-1} = (\tau\sigma_{\alpha_1}\tau^{-1})\cdots(\tau\sigma_{\alpha_r}\tau^{-1}) =\sigma_{\tau(\alpha_1)}\cdots\sigma_{\tau(\alpha_r)} \in\Ww ;
  \end{equation*}
  where the last equality is Lemma 9.2.
\end{proof}

\begin{ex}\label{9.7}
  Let $\alpha,\beta\in\Phi$ span a subspace $E'$ of $E$. Prove that $E'\cap\Phi$ is a root system in $E'$. Prove similarly that $\Phi\cap(\ZZ\alpha+\ZZ\beta)$ is a root system in $E'$ (must this coincide with $E'\cap \Phi$ ?). More generally, let $\Phi'$ be a nonempty subset of $\Phi$ such that $\Phi'=-\Phi'$, and such that $\alpha,\beta\in\Phi', \alpha+\beta\in\Phi$ implies $\alpha+\beta\in\Phi'$. Prove that $\Phi'$ is a root system in the subspace of $E$ it spans. [Use Table 1].
\end{ex}

\begin{ex}
  Compute root strings in $G_2$ to verify the relation $r - q =< \beta,\alpha >$.
\end{ex}

\begin{ex}\label{9.9}
  Let $\Phi$ be a set of vectors in a euclidean space $E$, satisfying only (R1), (R3), (R4). Prove that the only possible multiples of $\alpha\in\Phi$ which can be in $\Phi$ are $\pm\frac{1}{2}\alpha,\pm\alpha,\pm2\alpha$. Verify that $\{\alpha\in\Phi\mid 2\alpha\notin \Phi\}$ is a root system.
\end{ex}

\begin{ex}
  Let $\alpha,\beta\in\Phi$. Let the $\alpha-$string through $\beta$ be $\beta-r\alpha,\cdots,\beta+q\alpha$, and let the $\beta-$string through $\alpha$ be $\alpha-r'\beta,\cdots\alpha+q'\beta$. Prove that $\frac{q(r+1)}{(\beta,\beta)} = \frac{q'(r'+1)}{(\alpha,\alpha)}$.
\end{ex}

\begin{ex}
  Let $c$ be a positive real number. If $\Phi$ possesses any roots of squared length $c$, prove that the set of all such roots is a root system in the subspace of $E$ it spans. Describe the possibilities occurring in Figure 1.
\end{ex}

\section{Simple roots and Weyl group}



\begin{ex}
  Let $\codual{\Phi}$ be the dual system of $\Phi$, $\codual{\Delta} = \{\codual{\alpha}\mid\alpha\in\Delta\}$. Prove that $\codual{\Delta}$ is a base of $\codual{\Phi}$.[Compare Weyl chambers of $\Phi$ and $\codual{\Phi}$.]
\end{ex}

\begin{ex}\label{10.2}
  If $\Delta$ is a base of $\Phi$, prove that the set $(\ZZ\alpha + \ZZ\beta) \cap \Phi (\alpha\neq\beta\in\Delta)$ is a root system of rank $2$ in the subspace of $E$ spanned by $\alpha,\beta$ (cf. Exercise \ref{9.7}). Generalize to an arbitrary subset of $\Delta$.
\end{ex}

\begin{ex}
  Prove that each root system of rank $2$ is isomorphic to one of those listed in (9.3).
\end{ex}

\begin{ex}
  Verify the Corollary of Lemma 10.2A directly for $G_2$.
\end{ex}
\begin{proof}
  $a+b+b+b+a, a+b+b+b, a+b+b, a+b, a, b$.
\end{proof}

\begin{ex}\label{10.5}
  If $\sigma\in\Ww$ can be written as a product of $t$ simple reflections, prove that $t$ has the same parity as $l(\sigma)$.
\end{ex}
\begin{proof}
  It is enough to prove that if the identity is written as $t$ simple reflections, then $t$ is even. To see this, first note that the number of negative roots in the set $\sigma_1\cdots\sigma_t(\Delta)$ has the same parity of $t$. This follows by induction since each $\sigma_i=\sigma_{\alpha_i}$ fixes the sign of $\alpha_j$ if $\alpha_j\neq\alpha_i$, and changes the sign of $\alpha_i$. So if $1=\sigma_1\cdots\sigma_t$, then $t$ is even because $\Delta$ has no negative roots.
\end{proof}

\begin{ex}
  Define a function $sn \colon \Ww \to \{\pm1\}$ by $sn(\sigma) = (-1)^{l(\sigma)}$. Prove that $sn$ is a homomorphism (cf. the case $A_2$, where $\Ww$ is isomorphic to the symmetric group $\Sss_3$).
\end{ex}
\begin{proof}
  This is immediate from Exercise \ref{10.5}: given $\sigma,\tau\in\Ww$, we have that $l(\sigma\tau)=l(\sigma)+l(\tau) (\mod  2)$.
\end{proof}

\begin{ex}
  Prove that the intersection of ``positive'' open half-spaces associated with any basis $\gamma_1,\cdots,\gamma_l$ of $E$ is nonvoid. [If $\delta_i$ is the projection of $\gamma_i$ on the orthogonal complement of the subspace spanned by all basis vectors except $\gamma_i$, consider $\gamma=\sum r_i\delta_i$ when all $r_i > 0$.]
\end{ex}
\begin{proof}
  $(\gamma,\delta_j)=\sum r_i(\delta_i,\delta_j)=(\gamma_j,\delta_j)>0$.
\end{proof}

\begin{ex}
  Let $\Delta$ be a base of $\Phi$, $\alpha\neq\beta$ simple roots, $\Phi_{\alpha\beta}$ the rank $2$ root system in $E_{\alpha\beta} = \RR\alpha+\RR\beta$ (see Exercise \ref{10.2} above).
  The Weyl group $\Ww_{\alpha\beta}$ of $\Phi_{\alpha\beta}$ is generated by the restrictions $\tau_{\alpha}, \tau_{\beta}$ to $E_{\alpha\beta}$ of $\sigma_{\alpha}, \sigma_{\beta}$, and $\Ww_{\alpha\beta}$ may be viewed as a subgroup of $\Ww$. Prove that the ``length'' of an element of $\Ww_{\alpha\beta}$ (relative to $\tau_{\alpha},\tau_{\beta}$) coincides with the length of the corresponding element of $\Ww$.
\end{ex}

\begin{ex}
  Prove that there is a unique element $\sigma$ in $\Ww$ sending $\Phi^{+}$ to $\Phi^{-}$ (relative to $\Delta$). Prove that any reduced expression for $\sigma$ must involve all $\sigma_{\alpha} (\alpha\in\Delta)$. Discuss $l(\sigma)$.
\end{ex}
\begin{proof}
  Note that $-\Delta=\{-\alpha\mid\alpha\in\Delta\}$ is also a base for $\Phi$, so since $\Ww$ acts transitively on bases of $\Phi$ (Theorem 10.3), there is a $\sigma\in\Ww$ such that $\sigma(\Delta) = -\Delta$. Then $\sigma$ necessarily takes positive roots of $\Phi$ to negative roots of $\Phi$ (relative to $\Delta$). If $\tau\in\Ww$ also has this property, then $\sigma\tau$ takes a positive base to another positive base. By definition, two bases can be positive with respect to $\Delta$ only if they are equal, so since $\Ww$ acts simply transitively on bases, $\sigma\tau = 1$, so $\tau=\sigma$ because $\sigma$ has order $2$ (for the same reason just discussed). Hence $\sigma$ is unique.

  Let $\sigma = \sigma_{\alpha_1}\cdots\sigma_{\alpha_t}$ be a reduced expression for $\sigma (\alpha_i \in\Delta)$. Suppose $\beta\in\Delta$ is not in this expression. Since $\sigma_\alpha(\beta) = \beta-<\beta,\alpha>\alpha$, it is clear that $\sigma_{\alpha_s}\cdots\sigma_{\alpha_t}$ cannot take $\beta$ to another simple root. Since each $\sigma_{\alpha_i}$ permutes $\Phi^{+}\setminus\{\alpha_i\}$ (Lemma 10.2B), $\sigma(\beta) \notin \Phi^{-}$, which is a contradiction.
  Hence a reduced expression for $\sigma$ must involve all $\sigma_{\alpha} (\alpha\in\Delta)$.

  Since $\sigma(\Phi^{+})=\Phi^{-}$, $l(\sigma) = n(\sigma) = \#(\Phi^{+}) = \#(\Phi)/2$.
\end{proof}

\begin{ex}
  Given $\Delta = \{\alpha_1,\cdots,\alpha_l\}$ in $\Phi$, let $\lambda = \sum\limits_{i=1}^l k_i\alpha_i$ ($k_i \in\ZZ$, all $k_i \geqslant 0$ or all $k_i \leqslant 0$). Prove that either $\lambda$ is a multiple (possibly $0$) of a root, or else there exists $\sigma\in\Ww$ such that $\sigma\lambda = \sum\limits_{i=1}^l k'_i\alpha_i$, with some $k'_i > 0$ and some $k'_i < 0$.
\end{ex}
\begin{proof}
  If $\lambda$ is not a multiple of any root, then the hyperplane $P_{\lambda}$ orthogonal to $\lambda$ is not included in $\bigcup\limits_{\alpha\in\Phi} P_{\alpha}$. Take $\mu\in P_{\lambda} - \bigcup\limits_{\alpha\in\Phi} P_{\alpha}$. Then find $\sigma\in\Ww$ for which all $(\alpha,\sigma\mu) > 0$. It follows that $0 = (\lambda,\mu) = (\sigma\lambda,\sigma\mu) = \sum\limits_{i=1}^l k_i(\alpha_i,\sigma\mu)$.
\end{proof}

\begin{ex}\label{10.11}
  Let $\Phi$ be irreducible. Prove that $\codual{\Phi}$ is also irreducible. If $\Phi$ has all roots of equal length, so does $\codual{\Phi}$ (and then $\codual{\Phi}$ is isomorphic to $\Phi$). On the other hand, if $\Phi$ has two root lengths, then so does $\codual{\Phi}$; but if $\alpha$ is long, then $\codual{\alpha}$ is short (and vice versa). Use this fact to prove that $\Phi$ has a unique maximal short root (relative to the partial order $\prec$ defined by $\Delta$).
\end{ex}
\begin{proof}
  Since $\codual(\codual{\Phi}) = \Phi$, to prove that $\Phi$ irreducible implies $\codual{\Phi}$ irreducible, it is enough to prove that if $\Phi$ is reducible, then so is $\codual{\Phi}$. But this is obvious because $(\codual{\alpha}, \codual{\beta}) = 0$ if and only if $(\alpha,\beta) = 0$.

  Also, if all roots of $\Phi$ have the same length, then $(\alpha,\alpha)$ is a constant $C$ for $\alpha\in\Phi$, so $\codual{\beta} = \frac{2}{C} \beta$ for all $\beta \in\Phi$, which means all root lengths are the same in $\codual{\Phi}$. Multiplication by this nonzero scalar gives an isomorphism between $\Phi$ and $\codual{\Phi}$.

  If instead $\Phi$ has $2$ root lengths, then so does $\codual{\Phi}$. This must hold because if $\codual{\Phi}$ had one root length, then so would $\codual(\codual{\Phi}) = \Phi$. Since the length of $\codual{\beta}$ gets shorter the longer $\beta$ is, it is clear that short roots of $\Phi$ correspond to long roots of $\codual{\Phi}$, and vice versa. Finally, a maximal root of $\codual{\Phi}$ is long (Lemma 10.4D), so corresponds to a maximal short root of $\Phi$ (heights are preserved in passing to duals).
\end{proof}

\begin{ex}\label{10.12}
  Let $\lambda\in\CCc(\Delta)$. If $\sigma\lambda=\lambda$ for some $\sigma\in\Ww$, then $\sigma = 1$.
\end{ex}
\begin{proof}
  Note that $\Ww$ sends chambers of $\Phi$ to other chambers, so if $\sigma\lambda=\lambda$, then $\sigma$ fixes a chamber of $\Phi$. Then the exercise is a consequence of Theorem 10.3(e) and the fact that the set of chambers of $\Phi$ and the set of bases of $\Phi$ are isomorphic as $\Ww-$sets (10.1).
\end{proof}

\begin{ex}
  The only reflections in $\Ww$ are those of the form $\sigma_{\alpha}(\alpha\in\Phi)$. [A vector in the reflecting hyperplane would, if orthogonal to no root, be fixed only by the identity in $\Ww$.]
\end{ex}
\begin{proof}
  Let $\tau \in \Ww$ be some reflection. If the reflecting hyperplane of $\tau$ is not orthogonal to a root of $\Phi$, then let $\gamma$ be a vector in this hyperplane. Then $\gamma$ is contained in $\CCc(\Delta)$ for some base $\Delta$. By Exercise \ref{10.12}, the only element of $\Ww$ fixing $\gamma$ is the identity, so $\tau = 1$.
\end{proof}

\begin{ex}
  Prove that each point of $E$ is $\Ww-$conjugate to a point in the closure of the fundamental Weyl chamber relative to a base $\Delta$. [Enlarge the partial order on $E$ by defining $\mu\prec\lambda$ iff $\lambda-\mu$ is a nonnegative $\RR-$linear combination of simple roots. If $\mu\in E$, choose $\sigma\in\Ww$ for which $\lambda=\sigma\mu$ is maximal in this partial order.]
\end{ex}
\begin{proof}
  This is similar to the proof of Theorem 10.3(a). Here we replace a regular element $\gamma$ with any point of $E$. The difference is that $(\sigma(\gamma),\alpha)$ may be $0$ for some $\alpha$. This will imply only that $(\sigma(\gamma),\alpha)\geqslant0$ for all $\alpha \in\Delta$, which is to say that $\sigma(\gamma) \in \overline{\CCc(\Delta)}$.
\end{proof}

\section{Classification}


\begin{ex}
  Verify the Cartan matrices (Table 1).
\end{ex}

\begin{ex}
  Calculate the determinants of the Cartan matrices (using induction on $l$ for types $A_l - D_l$), which are as follows:
  \begin{equation*}
    A_l : l + 1;\ B_l : 2;\ C_l : 2;\ D_l : 4;\ E_6 : 3;\ E_7 : 2;\ E_8, F_4, G_2 : 1
  \end{equation*}
\end{ex}
\begin{proof}
  By expanding along the first row, we get that $\det A_l = 2\det A_{l-1} - \det A_{l-2}$ (the second matrix needs to again be expanded along the first column). Similar relations hold for $B_l, C_l$ and $D_l$.

  By inspection, $\det A_1 = 2$ and $\det A_2=3$, so by induction, $\det A_l = l + 1$.

  Also, $\det B_1 = \det B_2 = 2$, and $\det C_1 = \det C_2 = 2$, so $\det B_l = 2, \det C_l = 2$.

  Furthermore $D_2 = A_1 \times A_1$ and $D_3 = A_3$, so $\det D_2 = \det D_3 = 4$, which gives $\det D_l = 4$.

  The determinants of $E_6, E_7, E_8, F_4$ and $G_2$ can all be done via row reduction.
\end{proof}

\begin{ex}\label{11.3}
  Use the algorithm of (11.1) to write down all roots for $G_2$. Do the same for $C_3$ :
  \begin{equation*}
    \begin{pmatrix}
      2 & -1 & 0 \\
      -1 & 2 & -1 \\
      0 & -2 & 2 \\
    \end{pmatrix}
  \end{equation*}
\end{ex}
\begin{proof}
  Using the algorithm of (11.1), we start with simple roots for $G_2$, a short root $\alpha$, and a long root $\beta$. We know that $<\alpha, \beta> = -1$ and $<\beta, \alpha> = -3$.
  This means that we have a root string
  \begin{equation*}
    \{\beta, \beta +\alpha, \beta +2\alpha, \beta +3\alpha\}
  \end{equation*}
  Also, if $2\beta +k\alpha$ is to be a root, we need $r-q = <\beta +k\alpha, \beta> = 2-k$ to be negative, i.e., $k \geqslant 3$. This means $2\beta + 3\alpha$ is also a root, and we have listed all positive roots of $G_2$ (cf. p. 44).

  In the case of $C_3$, we have $3$ simple roots, $\alpha$ and $\beta$ (which are short), and $\gamma$ (which is long). We immediately see that $\{\alpha + \beta, \gamma + \beta, \gamma + 2\beta\}$ are roots. Since $<\alpha + \beta, \gamma> = -1$, we also get that $\alpha + \beta + \gamma$ is a root.
  Also, $< \gamma + 2\beta, \alpha> = -2$, so $\gamma + 2\beta + \alpha$ and $\gamma + 2\beta + 2\alpha$ are also roots. All other combinations of roots result in nonnegative brackets $<,>$, so these are all of the positive roots of $C_3$. To summarize, the positive roots are:
  \begin{equation*}
    \{\alpha, \beta, \gamma, \alpha + \beta, \beta + \gamma, 2\beta + \gamma, \alpha + \beta + \gamma, \alpha + 2\beta + \gamma, 2\alpha + 2\beta + \gamma\}
  \end{equation*}
\end{proof}

\begin{ex}
  Prove that the Weyl group of a root system $\Phi$ is isomorphic to the direct product of the respective Weyl groups of its irreducible components.
\end{ex}
\begin{proof}
  By induction on the number of components, we need only show this in the case that $\Phi$ is partitioned into two orthogonal components $\Phi_1$ and $\Phi_2$.
  Let $\Ww_1$ and $\Ww_2$ be their respective Weyl groups, and let $\Ww$ be the Weyl group of $\Phi$.

  Then $\Ww_1$ and $\Ww_2$ are commuting subgroups of $\Ww$ because of the orthogonality condition. Since $\Ww$ is generated by reflections in both $\Ww_1$ and $\Ww_2$, and $\Ww_1 \cap \Ww_2 = 1$, we see that $\Ww = \Ww1 \times \Ww2$.
\end{proof}

\begin{ex}
  Prove that each irreducible root system is isomorphic to its dual, except that $B_l, C_l$ are dual to each other.
\end{ex}
\begin{proof}
  Since $<\codual{\beta},\codual{\alpha}> = <\alpha,\beta>$ Exercise \ref{9.2}, the Cartan matrix of $\codual{\Phi}$ is the transpose of the Cartan matrix of $\Phi$. In terms of Dynkin diagrams, this corresponds to reversing the directions of the arrows. In the case of $A_l, D_l, E_6, E_7$ and $E_8$, nothing happens, so they are self-dual. In the cases of $F_4$ and $G_2$, one can find an isomorphism to their duals by reordering the simple roots. Finally, $B_l$ and $C_l$ become one another under this correspondence, so they are dual to one another.
\end{proof}

\begin{ex}\label{11.6}
  Prove that an inclusion of one Dynkin diagram in another (e.g., $E_6$ in $E_7$ or $E_7$ in $E_8$) induces an inclusion of the corresponding root systems.
\end{ex}
\begin{proof}
  An inclusion of Dynkin diagrams $D_1\injection D_2$ corresponds to the Cartan matrix of $D_1$ being a submatrix of the Cartan matrix of $D_2$.
\end{proof}

\section{Construction of root systems and automorphisms}

\begin{defn}
  Let $\Gamma=\{\sigma\in\Aut\Phi\mid\sigma(\Delta)=\Delta\}$, then $\Aut\Phi=\Ww\rtimes\Gamma$. This $\Gamma$ is usually viewed as the group of \termin{diagram automorphisms} or \termin{graph automorphisms}.
\end{defn}

\begin{exam}[$A_l(l\geqslant1)$]
  $E=\Span\{\varepsilon_1+\cdots+\varepsilon_{l+1}\}^{\perp}\subset\RR^{l+1}, I'=I\cap E$.

  $\Phi=\{\alpha\in I'\mid (\alpha,\alpha)=2\}=\{\varepsilon_i-\varepsilon_j, i\neq j\}$,
  $\Delta=\{\alpha_i=\varepsilon_i-\varepsilon_{i+1}, 1\leqslant i\leqslant l\}$.

  Weyl group: $\Ww\cong\Sss_{l+1}$ by $\sigma_{\alpha_i}\mapsto(i,i+1)$. $\Gamma\cong\ZZ/2$ when $l\geqslant2$.
\end{exam}

\begin{exam}[$B_l(l\geqslant2)$]
  $E=\RR^l$.
  $\Phi=\{\alpha\in I\mid (\alpha,\alpha)=1, 2\}=\{\pm\varepsilon_i, \pm(\varepsilon_i\pm\varepsilon_j), i\neq j\}$,

  $\Delta=\{\varepsilon_1-\varepsilon_2,\cdots,\varepsilon_{l-1}-\varepsilon_l,\varepsilon_l\}$.

  Weyl group: $\Ww\cong(\ZZ/2)^l\rtimes\Sss_l$ by corresponding $\sigma_{\varepsilon_i}$ to sign changes. $\Gamma=1$.
\end{exam}

\begin{exam}[$C_l(l\geqslant3)$]
  $E=\RR^l$. $C_l$ is dual to $B_l$, hence

  $\Phi=\{\pm2\varepsilon_i, \pm(\varepsilon_i\pm\varepsilon_j), i\neq j\}$,

  $\Delta=\{\varepsilon_1-\varepsilon_2,\cdots,\varepsilon_{l-1}-\varepsilon_l,2\varepsilon_l\}$.

  Weyl group: $\Ww\cong(\ZZ/2)^l\rtimes\Sss_l$ same as $B_l$. $\Gamma=1$.
\end{exam}

\begin{exam}[$D_l(l\geqslant4)$]
  $E=\RR^l$.
  $\Phi=\{\alpha\in I\mid (\alpha,\alpha)=2\}=\{\pm(\varepsilon_i\pm\varepsilon_j), i\neq j\}$,

  $\Delta=\{\varepsilon_1-\varepsilon_2,\cdots,\varepsilon_{l-1}-\varepsilon_l, \varepsilon_{l-1}+\varepsilon_l\}$.

  Weyl group: $\Ww\cong(\ZZ/2)^{l-1}\rtimes\Sss_l$
  where $\sigma_{\varepsilon_i+\varepsilon_j}\sigma_{\varepsilon_i-\varepsilon_j}$ is corresponding to the sign change of $i,j$ position.

  $\Gamma=\Sss_3$ when $l=4$, and $\ZZ/2$ when $l>4$.
\end{exam}

\begin{exam}[$E_6.E_7,E_8$]
  It suffices to construct $E_8$.

  $E=\RR^8, I'=I+\ZZ((\varepsilon_1+\cdots+\varepsilon_8)/2)$, $I''=$ subgroup of $I'$ consisting of all elements $\sum c_i\varepsilon_i+\frac{c}{2}(\varepsilon_1+\cdots+\varepsilon_8)$ for which $\sum c_i$ is an even integer.

  $\Phi=\{\alpha\in I''\mid (\alpha,\alpha)=2\}=\{\pm(\varepsilon_i\pm\varepsilon_j), i\neq j\}\cup\{\frac{1}{2}\sum(-1)^{k_i}\varepsilon_i\}$ (where the $k_i=0,1$, add up to an even integer).

  $\Delta=\{\frac{1}{2}(\varepsilon_1+\varepsilon_8-(\varepsilon_2+\cdots+\varepsilon_7)), \varepsilon_1+\varepsilon_2, \varepsilon_2-\varepsilon_1, \varepsilon_3-\varepsilon_2, \varepsilon_4-\varepsilon_3, \varepsilon_5-\varepsilon_4, \varepsilon_6-\varepsilon_5, \varepsilon_7-\varepsilon_6\}$.

  Weyl group has order $2^{14}3^55^27=696729600$.
\end{exam}

\begin{exam}[$F_4$]
  $E=\RR^4, I'=I+\ZZ((\varepsilon_1+\varepsilon_2+\varepsilon_3+\varepsilon_4)/2)$.

  $\Phi=\{\alpha\in I'\mid (\alpha,\alpha)=1, 2\}=\{\pm\varepsilon_i, \pm(\varepsilon_i\pm\varepsilon_j), i\neq j\}\cup\{\pm\frac{1}{2}(\varepsilon_1\pm\varepsilon_2\pm\varepsilon_3\pm\varepsilon_4)\}$.

  $\Delta=\{\varepsilon_2-\varepsilon_3, \varepsilon_3-\varepsilon_4, \varepsilon_4, \frac{1}{2}(\varepsilon_1-\varepsilon_2-\varepsilon_3-\varepsilon_4)\}$.

  Weyl group has order $1152$.
\end{exam}

\begin{exam}[$G_2$]
  $E=\Span\{\varepsilon_1+\varepsilon_2+\varepsilon_3\}^{\perp}\subset\RR^3, I'=I\cap E$.

  $\Phi=\{\alpha\in I'\mid (\alpha,\alpha)=2,6\}=\pm\{ \varepsilon_1-\varepsilon_2, \varepsilon_2-\varepsilon_3, \varepsilon_1-\varepsilon_3, 2\varepsilon_1-\varepsilon_2-\varepsilon_3, 2\varepsilon_2-\varepsilon_1-\varepsilon_3, 2\varepsilon_3-\varepsilon_1-\varepsilon_2 \}$.

  $\Delta=\{ \varepsilon_1-\varepsilon_2, -2\varepsilon_1+\varepsilon_2+\varepsilon_3 \}$.

  Weyl group: $\Ww\cong D_6$.
\end{exam}

\begin{ex}
  Verify the details of the constructions in (12.1).
\end{ex}

\begin{ex}
  Verify Table 2.
  \begin{center}
    \begin{tabular}{cll}
      \hline
      % after \\: \hline or \cline{col1-col2} \cline{col3-col4} ...
      Type & Long & Short \\
      \hline
      $A_l$ & $\alpha_1 + \alpha_2 + \cdots + \alpha_l$ & \\
      $B_l$ & $\alpha_1 + 2\alpha_2 + 2\alpha_3 + \cdots + 2\alpha_l$ & $\alpha_1 + \alpha_2 + \cdots + \alpha_l$ \\
      $C_l$ & $2\alpha_1 + 2\alpha_2 + \cdots + 2\alpha_{l-1} + \alpha_l$ & $\alpha_1 + 2\alpha_2 + \cdots + 2\alpha_{l-1} + \alpha_l$ \\
      $D_l$ & $\alpha_1 + 2\alpha_2 + \cdots + 2\alpha_{l-2} + \alpha_{l-1} + \alpha_l$ & \\
      $E_6$ & $\alpha_1 + 2\alpha_2 + 2\alpha_3 + 3\alpha_4 + 2\alpha_5 + \alpha_6$ & \\
      $E_7$ & $2\alpha_1 + 2\alpha_2 + 3\alpha_3 + 4\alpha_4 + 3\alpha_5 + 2\alpha_6 + \alpha_7$ & \\
      $E_8$ & $2\alpha_1 + 3\alpha_2 + 4\alpha_3 + 6\alpha_4 + 5\alpha_5 + 4\alpha_6 + 3\alpha_7 + 2\alpha_8$ & \\
      $F_4$ & $2\alpha_1 + 3\alpha_2 + 4\alpha_3 + 2\alpha_4$ & $\alpha_1 + 2\alpha_2 + 3\alpha_3 + 2\alpha_4$ \\
      $G_2$ & $3\alpha_1 + 2\alpha_2$ & $2\alpha_1 + \alpha_2$ \\
      \hline
    \end{tabular}
  \end{center}
\end{ex}

\begin{ex}
  Let $\Phi\subset E$ satisfy (R1), (R3), (R4), but not (R2), cf. Exercise \ref{9.9}. Suppose moreover that $\Phi$ is irreducible, in the sense of Section 11. Prove that $\Phi$ is the union of root systems of type $B_n, C_n$ in $E$ ($n = \dim E$), where the long roots of $B_n$ are also the short roots of $C_n$. (This is called the non-reduced root system of type $BC_n$ in the literature.)
\end{ex}

\begin{ex}
  Prove that the long roots in $G_2$ form a root system in $E$ of type $A_2$.
\end{ex}
\begin{proof}
  Let $\alpha$ be a short simple root of $G_2$, and let $\beta$ be a long simple root. The long positive roots of $G_2$ are $\{\beta, 3\alpha+\beta, 3\alpha+2\beta\}$ Exercise \ref{11.3}. It is clear from this description that the long roots form a root system, and that $\{\beta, -3\alpha -2\beta\}$ forms a base. Using the Cartan matrix for $G_2$, one deduces that the Cartan matrix for this base is the same as that of $A_2$.
\end{proof}

\begin{ex}
  In constructing $C_l$, would it be correct to characterize $\Phi$ as the set of all vectors in $I$ of squared length $2$ or $4$? Explain.
\end{ex}
\begin{proof}
  No, this would give vectors such as $\pm4\varepsilon_i$. But ignoring that problem, one would also have vectors like $2\varepsilon_1 + \varepsilon_2 + \varepsilon_3$. The resulting set of vectors would be much larger than $\Phi$.
\end{proof}

\begin{ex}\label{12.6}
  Prove that the map $\alpha\mapsto -\alpha$ is an automorphism of $\Phi$. Try to decide for which irreducible $\Phi$ this belongs to the Weyl group.
\end{ex}
\begin{proof}
  It is immediate that $\alpha\mapsto -\alpha$ is an automorphism of $\Phi$ since
  \begin{equation*}
    <-\alpha,-\beta> = \frac{2(-\alpha,-\beta)}{(-\beta,-\beta)} = \frac{2(\alpha, \beta)}{(\beta, \beta)} = <\alpha, \beta>.
  \end{equation*}

  Since $\Aut\Phi$ is a semidirect product of $\Ww$ and the subgroup $\Gamma$, it is immediate that $\alpha\mapsto -\alpha$ is an element of $\Ww$ for $B_l, C_l, E_7, E_8, F_4$ and $G_2$ by Table 12.1.

  For $A_l$, $\alpha\mapsto -\alpha$ is not an element of $\Ww$ if $l > 1$. To see this, we use the description of $A_l$ as the set of vectors $\{\varepsilon_i -\varepsilon_j, i\neq j\}$ in $\RR^{l+1}$. Then the reflections $\sigma_{\alpha_i}$ where $\alpha_i = \varepsilon_i - \varepsilon_{i+1}$ permutes the vectors $\varepsilon_i$ and $\varepsilon_{i+1}$ and leaves the other standard vectors fixed. From this, it is clear that, for example, one cannot send $(\varepsilon_1 - \varepsilon_2, \varepsilon_1 - \varepsilon_3)$ to $(\varepsilon_2 - \varepsilon_1, \varepsilon_3 - \varepsilon_1)$ because this would require a permutation which swapped $1$ with $2$ and swapped $1$ with $3$.
  Of course it is clear that $A_1$ has $\alpha\mapsto -\alpha$ as an element of its Weyl group.

  Similarly, $D_l$ for $l \geqslant 4$ does not have $\alpha\mapsto -\alpha$ in its Weyl group for $l$ odd, but does for $l$ even. It can be described as the set of vectors $\{\pm\varepsilon_i \pm \varepsilon_j i\neq j\}$ in $\RR^l$, and its Weyl group consists of permutations of the $\varepsilon_i$ along the sign changes that involve an even number of sign changes.

  Not sure about $E_6$.
\end{proof}

\begin{ex}
  Describe $\Aut\Phi$ when $\Phi$ is not irreducible
\end{ex}
\begin{proof}
  Write $\Phi = \Phi_1 \cup \cdots \cup \Phi_r$ (disjoint) where the $\Phi_i$ are irreducible root systems and $(\Phi_i,\Phi_j) = 0$. Any automorphism $\sigma$ of $\Phi$ must satisfy $\sigma(\Phi_i) \subset \Phi_j$ because irreducibility is an invariant of isomorphism of root systems, i.e., if $\sigma(\Phi_i)$ were contained in two or more components of $\Phi$, then we could write it as a disjoint union of pairwise orthogonal sets. Since $\Phi$ is finite, it follows from a counting argument that $\sigma(\Phi_i) = \Phi_j$ . Let $S$ be the subgroup of permutations $\sigma$ of $\{1,\cdots. r\}$ such that $\Phi_i$ is isomorphic to $\Phi_{\sigma(i)}$. Then $\Aut\Phi$ is the semidirect product of $S$ with $\Aut\Phi_1 \times \cdots \times \Aut\Phi_r$.
\end{proof}

\section{Abstract theory of weights}

\begin{exam}[$\sl(3,F)$]
  $h_1=e_{11}-e_{22}, h_2=e_{22}-e_{33}, e_{12}, e_{13}, e_{23}, e_{21}, e_{31}, e_{32}$.
  \begin{equation*}
    \ad h_1= \diag(0,0,2,1,-1,-2,-1,1), \ad h_2=\diag(0,0,-1,1,2,1,-1,-2)
  \end{equation*}
  \begin{equation*}
  \alpha_1=(2,-1), \alpha_1+\alpha_2=(1,1), \alpha_2=(-1,2)
  \end{equation*}
  
  The Cartan matrix is $\begin{pmatrix}
                          2 & -1 \\
                          -1 & 2 \\
                        \end{pmatrix}$.
  
  \begin{align*}
    &\alpha_1=2\lambda_1-\lambda_2, \alpha_2=-\lambda_1+2\lambda_2;\\
    &\lambda_1=\frac{1}{3}(2\alpha_1+\alpha_2), \lambda_2=\frac{1}{3}(\alpha_1+2\alpha_2)
  \end{align*}
  
  $\ord(\lambda_1)=\ord(\lambda_2)=3$, the fundamental group is Cycle$(3)$.
  
  \begin{equation*}
    \delta=\frac{1}{2}(\alpha_1+\alpha_2+\alpha_1+\alpha_2)=(1,1)=\lambda_1+\lambda_2
  \end{equation*}
  
  The standard set with highest weight must be like the follow:
  \begin{equation*}
    \Pi=\{m,m-2,,\cdots,-m\}
  \end{equation*}
  with highest weight $m$.
\end{exam}

\begin{ex}
  Let $\Phi = \Phi_1 \cup \cdots \cup \Phi_t$ be the decomposition of $\Phi$ into its irreducible components, with $\Delta = \Delta_1 \cup \cdots \cup \Delta_t$. Prove that $\Lambda$ decomposes into a direct sum $\Lambda_1 \oplus \cdots \oplus \Lambda_t$; what about $\Lambda^{+}$?
\end{ex}
\begin{proof}
  Given any element $\lambda \in \Lambda$, let $\lambda_i$ be defined by $<\lambda_i,\alpha> = <\lambda, \alpha>$ if $\alpha \in\Phi_i$ and $0$ otherwise. In other words, let $\lambda_i$ be the orthgonal projection of $\lambda$ onto the subspace spanned by $\Phi_i$. Then $\lambda = \lambda_1 + \cdots + \lambda_t$, so $\Lambda = \Lambda_1 + \cdots + \Lambda_t$. It is clear that $\Lambda_i \cap \Lambda_j = 0 $if $i \neq j$, so the sum is direct.
  
  However, we cannot say the same thing about $\Lambda^{+}$. For example, consider the root system $A_1\times A_1$ with base $\{(1, 0), (0, 1)\}$. Write $A_1 \times A_1 = \Phi_1 \cup \Phi_2$ where $\Phi_1 = \{(\pm1, 0)\}$ and $\Phi_2 = \{(0,\pm1)\}$. Then $(2,-1) \in \Lambda^{+}$, but $(0,-1) \notin \Lambda^{+}_2$, so we do not have $\Lambda^{+}$ generated by $\Lambda^{+}_1$ and $\Lambda^{+}_2$.
\end{proof}

\begin{ex}
  Show by example (e.g., for $A_2$) that $\lambda\notin\Lambda^{+}, \alpha\in\Delta, \lambda-\alpha\in\Lambda^{+}$  is possible.
\end{ex}
\begin{proof}
  In $A_2$, $\lambda=3\lambda_1-\lambda_2\notin\Lambda^{+}, \alpha=2\lambda_1-\lambda_2\in\Delta, \lambda-\alpha=\lambda_1\in\Lambda^{+}$.
\end{proof}

\begin{ex}
  Verify some of the data in Table 1, e.g., for $F_4$.
\end{ex}

\begin{ex}
  Using Table 1, show that the fundamental group of $A_l$ is cyclic of order $l + 1$, while that of $D_l$ is isomorphic to $\ZZ/4$ ($l$ odd), or $\ZZ/2\times \ZZ/2$ ($l$ even). (It is easy to remember which is which, since $A_3 = D_3$.)
\end{ex}
\begin{proof}
  The first is clear because the element $\lambda_i$ listed for $A_l$ in Table 1 does not have order smaller than $l+ 1$. If it did, then $\frac{j(l-i+1)}{l+1}$ would be an integer for $j < l+ 1$ which is not true for $i = 1$, for example. Since the determinant of the Cartan matrix for $A_l$ is $l+ 1$, we conclude that its fundamental group is $\ZZ/(l+ 1)$.
  
  The same considerations show that if $l$ is even, then every element $\lambda_i$ listed for $D_l$ has order $2$, whereas if $l$ is odd, then $\lambda_l$ has order $4$. Hence the respective fundamental groups are $\ZZ/2\times \ZZ/2$ for $l$ even, and $\ZZ/4$ for $l$ odd since the Cartan matrix for $D_l$ has determinant $4$.
\end{proof}

\begin{ex}
  If $\Lambda'$ is any subgroup of $\Lambda$ which includes $\Lambda_r$, prove that $\Lambda'$ is $\Ww-$invariant. Therefore, we obtain a homomorphism $\phi\colon \Aut\Phi/\Ww \to \Aut(\Lambda/\Lambda_r)$. Prove that $\phi$ is injective, then deduce that $ -1\in \Ww$ if and only if $\Lambda_r \supset 2\Lambda$ (cf. Exercise \ref{12.6}). Show that $ -1\in\Ww$ for precisely the irreducible root systems $A_1,B_l,C_l,D_l$($l$ even), $E_7, E_8,  F_4, G_2$.
\end{ex}
\begin{proof}
  $\sigma_I\lambda_j=\lambda_j-\delta_{ij}\alpha_i$, hence $\sigma(\lambda+\Lambda_r)=\lambda+\Lambda_r$. Then $\Lambda'$ is $\Ww-$invariant since $\Lambda'=\coprod\limits_{\lambda}\lambda+\Lambda_r$.
  \begin{equation*}
    \ker\phi=\{\sigma\mid \sigma(\lambda+\Lambda_r)=\lambda+\Lambda_r, \forall \lambda\in\Lambda\}\quad \Aut\Phi/\Ww=\Gamma
  \end{equation*}
  
\end{proof}

\begin{ex}
  Prove that the roots in $\Phi$ which are dominant weights are precisely the highest long root and (if two root lengths occur) the highest short root (cf. (10.4) and Exercise \ref{10.11}), when $\Phi$ is irreducible.
\end{ex}
\begin{proof}
	By Lemma 10.4C, roots of same length are $\mathcal{W}-$conjugate to exactly one dominant weight. By Lemma 13.2A, the dominant weight is maximal among its $\mathcal{W}-$orbit. As $\mathcal{W}$ permutes the roots of same length, the conclusion follows.
\end{proof}

\begin{ex}\label{13.7}
  If $\varepsilon_1, \cdots, \varepsilon_l$ is an \termin{obtuse} basis of the euclidean space $E$ (i.e., all $(\varepsilon_i, \varepsilon_j) \leqslant 0$ for $i \neq j$), prove that the dual basis is \termin{acute} (i.e., all $(\varepsilon^{\ast}_i, \varepsilon^{\ast}_j) \geqslant 0$ for $i \neq j$). [Reduce to the case $l = 2$.]
\end{ex}
\begin{proof}
	In the case $l=2$, this is obvious. Let's show this by induction on $l$. Then, for an euclidean space $E$ with obtuse basis $\varepsilon_1, \cdots, \varepsilon_{l+1}$. Then let $E'$ be the subspace spanned by $\varepsilon_1, \cdots, \varepsilon_l$. Let $\varepsilon'_1, \cdots, \varepsilon'_l$ be the dual basis in $E'$ which, by the conduction hypothesis, is acute. Then the dual basis of $\varepsilon_1, \cdots, \varepsilon_{l+1}$ satisfies
	\[\varepsilon^{\ast}_i = \varepsilon'_i - (\varepsilon'_i,\varepsilon_{l+1})\varepsilon^{\ast}_{l+1},\qquad i=1,\cdots,l.\]
	Therefore $(\varepsilon^{\ast}_i,\varepsilon^{\ast}_j) = (\varepsilon'_i,\varepsilon'_j)\geqslant0$ for $1\leqslant i,j \leqslant l$. On the other hand, since $\varepsilon'_i$ are linear combinations of $\varepsilon_1, \cdots, \varepsilon_l$ with nonnegative coefficients, we conclude that $(\varepsilon^{\ast}_i,\varepsilon^{\ast}_{l+1}) \geqslant0$. Note that this proof actually works for more strict result.
\end{proof}

\begin{ex}
  Let $\Phi$ be irreducible. Without using the data in Table 1, prove that each $\lambda_i$ is of the form $\sum \limits_j q_{ij}\alpha_j$, where all $q_{ij}$ are positive rational numbers. [Deduce from Exercise \ref{13.7} that all $q_{ij}$ are nonnegative. From $(\lambda_i, \lambda_i) > 0$, show that $q_{ii}>0$. Then show that if $q_{ij} > 0$ and $(\alpha_j, \alpha_k) < 0$, then $q_{ik} > 0$.]
\end{ex}
\begin{proof}
	By Lemma 10.1, $\alpha_i$ form an obtuse basis. Therefore $(\lambda_i, \lambda_j) \geqslant 0$. Since
	\[
	(\lambda_i, \lambda_j) = (\sum_k q_{ik}\alpha_k, \lambda_j) = \sum_{k}q_{ik} (\alpha_k,\lambda_j) = \frac{q_{ij}}{2}(\alpha_j,\alpha_j),
	\]
	We have $q_{ij}\geqslant 0$. In particular, since $(\lambda_i, \lambda_i) > 0$, $q_{ii}>0$. Since $\Phi$ is irreducible, the Coxeter graph is connected. Thus for any $\alpha_j$, there exists some $\alpha_k$ such that $(\alpha_j, \alpha_k) < 0$. Follow a similar reasoning of Exercise \ref{13.7}, we see that all $(\lambda_i, \lambda_j)$ are positive. 
\end{proof}

\begin{ex}
  Let $\lambda\in \Lambda^{+}$. Prove that $\sigma(\lambda + \delta) - \delta$ is dominant only for $\sigma = 1$.
\end{ex}
\begin{proof}
	Since $\delta$ is strongly dominant, we have
	\[\delta\succ\sigma(\delta),\qquad\delta\succ\sigma^{-1}(\delta),\]
	and the equality holds only for $\sigma = 1$.
	Hence $(\sigma(\lambda + \delta) - \delta)\prec(\lambda + \delta - \sigma^{-1}(\delta))$ and the equality holds only for $\sigma = 1$.
\end{proof}

\begin{ex}
  If $\lambda\in \Lambda^{+}$, prove that the set $\Pi$ consisting of all dominant weights $\mu\prec\lambda$ and their $\Ww-$conjugates is saturated, as asserted in (13.4).
\end{ex}

\begin{ex}
  Prove that each subset of $\Lambda$ is contained in a unique smallest saturated set, which is finite if the subset in question is finite.
\end{ex}

\begin{ex}
  For the root system of type $A_2$, write down the effect of each element of the Weyl group on each of $\lambda_1, \lambda_2$. Using this data, determine which weights belong to the saturated set having highest weight $\lambda_1 + 3\lambda_2$. Do the same for type $G_2$ and highest weight $\lambda_1 + 2\lambda_2$.
\end{ex}

\begin{ex}
  Call $\lambda \in \Lambda^{+}$ \termin{minimal} if $\mu \in \Lambda^{+}, \mu\prec \lambda$ implies that $\mu = \lambda$. Show that each coset of $\Lambda_r$ in $\Lambda$ contains precisely one minimal $\Lambda$. Prove that $\lambda$ is minimal if and only if the $\Ww-$orbit of $\lambda$ is saturated (with highest weight $\lambda$), if and only if $\lambda \in \Lambda^{+}$ and $< \lambda, \alpha >= 0, 1, -1$ for all roots $\alpha$. Determine (using Table 1) the nonzero minimal $\lambda$ for each irreducible $\Phi$, as follows:
  \begin{align*}
    A_l &\colon\lambda_1, \cdots , \lambda_l \\
    B_l &\colon\lambda_l \\
    C_l &\colon\lambda_1 \\
    D_l &\colon\lambda_1, \lambda_{l-1}, \lambda_l \\
    E_6 &\colon\lambda_1, \lambda_6 \\
    E_7 &\colon\lambda_7
  \end{align*}
\end{ex}
